\documentclass[12pt, Times New Roman]{article}
\usepackage{amsmath}
\usepackage{babel}
\usepackage{graphicx}
\usepackage{listings}
\usepackage{xcolor}


\title{Space Shuttle Main Engine}

\definecolor{codegreen}{rgb}{0,0.6,0}
\definecolor{codegray}{rgb}{0.5,0.5,0.5}
\definecolor{codepurple}{rgb}{0.58,0,0.82}
\definecolor{backcolour}{rgb}{0.95,0.95,0.92}

\lstdefinestyle{mystyle}{
    backgroundcolor=\color{backcolour},   
    commentstyle=\color{codegreen},
    keywordstyle=\color{magenta},
    numberstyle=\tiny\color{codegray},
    stringstyle=\color{codepurple},
    basicstyle=\ttfamily\footnotesize,
    breakatwhitespace=false,         
    breaklines=true,                 
    captionpos=b,                    
    keepspaces=true,                 
    numbers=left,                    
    numbersep=5pt,                  
    showspaces=false,                
    showstringspaces=false,
    showtabs=false,                  
    tabsize=2
}

\lstset{style=mystyle}


\begin{document}

    \maketitle
    
    \tableofcontents

    \section{Introduction}
    The Space Shuttle Main Engine (SSME) is a liquid-fueled rocket engine used to power the Space Shuttle Orbiter. 
    This report will discuss the design and methodologies of the SSME code, as well as the results of the code.
    For this project, we were given the parameters below:
    \begin{center}
        \begin{tabular}{|c|c|}
            \hline
            Combustion Chamber Data & \\
            \hline
            Fuel & Hydrogen \\
            \hline
            Oxidizer & $O_{2}$ \\
            \hline
            Fuel/Oxidizer Ratio & 0.166 \\
            \hline
            Hydrogen Injection Temperature & 850 K \\
            \hline
            Oxidizer Injection Temperature & 530 K \\
            \hline
            Chamber Pressure & 204 atm \\
            \hline
        \end{tabular}

        \vspace*{6pt}

        \begin{tabular}{|c|c|}
            \hline
            $R_{t}$ & 5.15 inch \\
            \hline
            $R_{e}$ & 45.35 inch \\
            \hline
            $\theta_{p}$ & 32 degree \\
            \hline
        \end{tabular}
    \end{center}

    \section{Oxygen}

    \subsection{First Enthalpy Exchanger}
    We begin with computing the equivalence ratio $\phi$  for the oxygen first enthalpy exchanger. 
    The equivalence ratio is defined as the ratio of the mass of fuel to the mass of oxidizer. 
    The mass of fuel is the mass of hydrogen, and the mass of oxidizer is the mass of oxygen. 
    The mass of hydrogen is $\phi \frac{mol_{O_{2}}}{mol_{H_{2}}} = \Phi$.


    \vspace*{6pt}

    From here we are to calculate the stndard heat of formation by:
    
    \begin{equation}
        \Delta \hat{h}_{H_{2}} = \hat{h}_{H_{2}}(850) \hat{R}T^{2}\int^{pi}_{pf} \frac{1}{p} \frac{-\hat{b}_{H_{2}}p}{\hat{R}T^{2}} dp
    \end{equation}

    \vspace*{6pt}

    \begin{equation}
        \Delta \hat{h}_{O_{2}} = \hat{h}_{O_{2}}(530) + \hat{R}T^{2}\int^{pi}_{pf} \frac{1}{p} \frac{-\hat{b}_{O_{2}}p}{\hat{R}T^{2}} dp
    \end{equation}

    \begin{center}
        \begin{tabular}{|c|c|}
            \hline
            Species & $\hat{b}$ \\
            \hline
            $H_{2}$ & 0.0266 \\
            \hline
            $O_{2}$ & 0.0318 \\
            \hline
            $H_{2}O$ & 0.03049 \\
            \hline
            $N_{2}$ & 0.0391 \\
            \hline
            $F_{2}$ & 0.02896 \\
            \hline
            $HF$ & 0.0739 \\
            \hline
        \end{tabular}
    \end{center}
    
    Where $\hat{b}$ is read from a table developed in \verb|exchanger.py| file. $\Delta \hat{h}_{i}$ is calculated from both 
    hydrogen and oxygen. Once we calculate $\Delta \hat{h}_{i}$ for each, we can calculate the enthalpy of the mixture.

    \begin{equation}
        \Delta \hat{h}_{mix} = \Phi \Delta \hat{h}_{H_{2}} + \Delta \hat{h}_{O_{2}}
    \end{equation}
    
    We can then apply this to our chemical equation.

    \begin{equation}
        O_{2} + \Phi H_{2} \rightarrow (2-\Phi)H_{2} +2H_{2}O
    \end{equation}

    Note: notice the $(2-\Phi)$ in the equation above. This is because we calculated the enthalpy of formation for hydrogen and not 
    hydrogen gas $H_{2}$.
    
    \vspace*{6pt}

    From here we can sum the molar enthalpies of formation to get the first estimate of the first reaction enthalpy

    \begin{equation}
        \Delta \hat{h}_{r} = \Phi \Delta \hat{h}_{H_{2}} + 2(\Delta \hat{h}_{H_{2}O})
    \end{equation}

    From here we go into our Temperature finding Algorithm. We are given the following parameters:

    \begin{equation}
        r = \Phi - 1
    \end{equation}

    \begin{equation}
        q = 1 
    \end{equation}

    \begin{equation}
        r_{percent} = \frac{r}{r+q}
    \end{equation}

    \begin{equation}
        q_{percent} = \frac{q}{r+q}
    \end{equation}

    From here we use the values of (Eq. 6) - (Eq. 9) to calculate the temperature of the first enthalpy exchanger.

    \begin{equation}
        T_{H_{2}} = \verb|h2_tableT|(\Delta \hat{h}_{r})
    \end{equation}

    \begin{equation}
        T_{H_{2}O} = \verb|ho2_tableT|(q_{percent} \Delta \hat{h}_{r})
    \end{equation}

    Once we have returned the values of Temperature, we then average the two values to get the final temperature of the first enthalpy exchanger.

    \begin{equation}
        T_{C} = \frac{T_{H_{2}} + T_{H_{2}O}}{2}
    \end{equation}

    \vspace*{6pt}

    \subsection{Combustion Chamber}
    For the Combustion Chamber, we have the two equations below:

    \begin{equation}
        H_{2}O \rightarrow OH + H 
    \end{equation}

    \begin{equation}
        H_{2} \rightarrow 2H
    \end{equation}

    The Gibbs Free Energy for each species is given by:
    
    \begin{equation}
        \hat{g}_{H_{2}O} = \verb|h2o_table_g|(T_{C})
    \end{equation}

    \begin{equation}
        \hat{g}_{H_{2}} = \verb|h2_table_g|(T_{C})
    \end{equation}

    \begin{equation}
        \hat{g}_{OH} = \verb|oh_table_g|(T_{C})
    \end{equation}

    \begin{equation}
        \hat{g}_{H} = \verb|h_table_g|(T_{C})
    \end{equation}

    \vspace*{6pt}

    We can then calculate the Gibbs Free Energy of the reaction by:

    \begin{equation}
        \Delta \hat{g}_{r} = 2(\hat{g}_{H} - \hat{g}_{H_{2}}) 
    \end{equation}

    \begin{equation}
        \Delta \hat{g}_{r} = \hat{g}_{H} - \hat{g}_{OH} + \hat{g}_{H_{2}O}
    \end{equation}

    Now is the point where we involve the \textbf{Law of Mass Action}:

    \begin{center}
        \begin{tabular}{|c|c|c|c|}
            \hline
            Species & $\nu_{(i)} '$ & $\nu_{(i)} ''$ & $\nu_{(i)} '' - \nu_{(i)} ''$ \\
            \hline
            $H_{2}O$ & 1 & 0 & -1 \\
            \hline
            $OH$ & 0 & 1 & 1 \\
            \hline
            $H$ & 0 & 1 & 1 \\
            \hline
        \end{tabular}
    \end{center}

    We an now use:

    \begin{equation}
        K_{n} = \prod^{N_{s}}_{i=1} X_{(n)}^{(\nu_{(i)} '' - \nu_{(i)} ')} = (\frac{p}{p_{a}})^{-\sigma_{v}}K_{p}
    \end{equation}

    Where;

    \begin{equation}
        \sigma_{v} = \sum^{N_{s}}_{i=1} (\nu_{(i)} '' - \nu_{(i)} ')
    \end{equation}

    \begin{equation}
        K_{p} = e^{- \frac{\Delta G}{\hat{R}T}}
    \end{equation}

    \begin{equation}
        \Delta G = \sum^{N_{s}}_{i=1} (\nu_{(i)} '' \hat{g}_{(i)}) \hat{g}_{(i)}
    \end{equation}

    From here we can solve for the equilibrium constant $K_{n}$.

    \begin{equation}
        K_{H_{2}O} = (\frac{204 atm}{1 atm})^{-\frac{\Delta G_{H_{2}O}}{\hat{R}T_{C}}}
    \end{equation}

    \begin{equation}
        K_{H_{2}} = (\frac{204 atm}{1 atm})^{-\frac{\Delta G_{H_{2}}}{\hat{R}T_{C}}}
    \end{equation}

    Now we move into the chemistry problem. 

    \textbf{Atom Balance}:

    \begin{equation}
        H_{2} + l O_{2} \rightarrow N_{H_{2}O} H_{2}O + N_{H_{2}} H_{2} + N_{H}H + N_{OH}OH
    \end{equation}

    Recall, due to the fule being Hydrogen;

    \begin{equation}
        mass_{fuel} = 2.02 
    \end{equation}

    \begin{equation}
        mass_{Oxidizer} =  \frac{mass_{fuel}}{\phi}
    \end{equation}

    \begin{equation}
        l = \frac{mass_{Oxidizer}}{2 \hat{m}_{O_{2}}}
    \end{equation}

    Where $\hat{m}_{O_{2}}$ is the molar mass of Oxygen. And $\phi$ is the Fuel to Oxidizer ratio, and not the equivalence ratio.
    Revisiting the \textbf{Atom Balance} equation, we can now solve for the number of moles of each species. 

    For $H_{2}O \rightarrow OH + H$:

    \begin{equation}
        H_{2}: 2 = 2N_{H_{2}O} + 2N_{H_{2}} + N_{H} + N_{OH}
    \end{equation}

    \begin{equation}
        O_{2}: 2l = N_{H_{2}O} + N_{OH}
    \end{equation}

    \begin{equation}
        1 = N_{H_{2}O} + N_{H_{2}} + \frac{1}{2}N_{H} + \frac{1}{2}N_{OH}
    \end{equation}

    Where $\frac{1}{2}N_{H} + \frac{1}{2}N_{OH} \rightarrow 0$. Resulting in:

    \begin{equation}
        1 = N_{H_{2}O} + N_{H_{2}}
    \end{equation}

    \begin{equation}
        2(0.38050899) = N_{H_{2}O} + N_{OH}
    \end{equation}

    Where $N_{OH} \rightarrow 0$. Resulting in:

    \begin{equation}
        0.760592 = N_{H_{2}O}
    \end{equation}

    \begin{equation}
        X = \frac{N_{Molecule}}{N_{Total}}
    \end{equation}

    \begin{equation}
        X_{H_{2}O} = \frac{N_{H_{2}O}}{N_{Total}}
    \end{equation}

    \begin{equation}
        X_{H} = \frac{N_{H}}{N_{Total}}
    \end{equation}

    \begin{equation}
        X_{OH} = \frac{N_{OH}}{N_{Total}}
    \end{equation}
        
    \begin{equation}
        X_{H_{2}O} = \frac{N_{H_{2}}}{N_{H_{2}O} + N_{H_{2}} + N_{H} + N_{OH}}
    \end{equation}

    \begin{equation}
        X_{H} = \frac{N_{H}}{N_{H_{2}O} + N_{H_{2}} + N_{H} + N_{OH}}
    \end{equation}

    \begin{equation}
        X_{OH} = \frac{N_{OH}}{N_{H_{2}O} + N_{H_{2}} + N_{H} + N_{OH}}
    \end{equation}

    Recall, $N_{OH} \rightarrow 0$ and $N_{H} \rightarrow 0$. Resulting in:

    \begin{equation}
        K_{n} = \frac{\frac{N_{OH}}{N_{H_{2}O + N_{H_{2}}}} \frac{N_{H}}{N_{H_{2}O} + N_{H_{2}}}}{\frac{N_{H_{2}O}}{N_{H_{2}O} + N_{H_{2}}}}
    \end{equation}

    Resulting in,

    \begin{equation}
        K_{H_{2}O} = \frac{N_{H} N_{OH}}{(N_{H} + N_{H_{2}O}) N_{H_{2}O}}
    \end{equation}

    Similarly we follow the same steps for the $H_{2} \rightarrow H + H$ reaction.

    \begin{equation}
        X_{H_{2}} = \frac{N_{H_{2}}}{Total}
    \end{equation}

    \begin{equation}
        X_{H} = \frac{N_{H}}{Total}
    \end{equation}

    \begin{equation}
        X_{H_{2}} = \frac{N_{H_{2}}}{N_{H_{2}} + N_{H_{2}O} + N_{H} + N_{OH}}
    \end{equation}

    \begin{equation}
        X_{H} = \frac{N_{H}}{N_{H_{2}} + N_{H_{2}O} + N_{H} + N_{OH}}
    \end{equation}

    \begin{equation}
        K_{H_{2}} = \frac{(\frac{N_{H}}{N_{H_{2}O} + N_{H_{2}}})^{2}}{\frac{N_{H_{2}}}{N_{H_{2}O} + N_{H_{2}}}}
    \end{equation}

    \begin{equation}
        K_{H_{2}} = \frac{N_{H}^{2}}{(N_{H_{2}O} + N_{H_{2}}) N_{H_{2}}}
    \end{equation}

    As a result we have 4 coupled equations:

    \begin{equation}
        1 = N_{H_{2}O} + N_{H_{2}}
    \end{equation}

    \begin{equation}
        0.760592 = N_{H_{2}O}
    \end{equation}

    \begin{equation}
        K_{H_{2}O} = \frac{N_{H} N_{OH}}{(N_{H} + N_{H_{2}O}) N_{H_{2}O}}
    \end{equation}

    \begin{equation}
        K_{H_{2}} = \frac{N_{H}^{2}}{(N_{H_{2}O} + N_{H_{2}}) N_{H_{2}}}
    \end{equation}

    Where $K_{H_{2}O}$ and $K_{H_{2}}$ are calculated values.
    
    Solving these equations we get:

    \begin{equation}
        N_{H_{2}O} = 0.760592
    \end{equation}

    \begin{equation}
        N_{H_{2}} = 0.239408
    \end{equation}

    \begin{equation}
        N_{H} = 0.05333
    \end{equation}

    \begin{equation}
        N_{OH} = 0.101958
    \end{equation}

    \begin{center}
        \begin{tabular}{|c|c|c|c|c|c|c|c|}
            \hline
            Species & $N_{(i)}$ & $X_{(i)}$ & $m_{(i)}$ &  $Y_{(i)}$ & $\hat{c}_{p(i)}$ & $c_{p(i)}$ & $Y_{(i)}c_{p}$ \\
            \hline
            $H_{2}O$ & $N_{H_{2}O}$ & $X_{H_{2}O} = N_{H_{2}O}$ & $\hat{m}N_{H_{2}O}$ & $\frac{\hat{m}}{m_{(i)}}$ & $\hat{c}_{p(H_{2}O)}(T_{C})$ & $\frac{\hat{c}_{p}}{\hat{m}}$ & $Y_{H_{2}O}c_{p}$ \\
            \hline
            $H_{2}$ & $N_{H_{2}}$ & $X_{H_{2}} = N_{H_{2}}$ & $\hat{m}N_{H_{2}}$ & $\frac{\hat{m}}{m_{(i)}}$ & $\hat{c}_{p(H_{2})}(T_{C})$ & $\frac{\hat{c}_{p}}{\hat{m}}$ & $Y_{H_{2}}c_{p}$ \\
            \hline
            $H$ & $N_{H}$ & $X_{H} = N_{H}$ & $\hat{m}N_{H}$ & $\frac{\hat{m}}{m_{(i)}}$ & $\hat{c}_{p(H)}(T_{C})$ & $\frac{\hat{c}_{p}}{\hat{m}}$ & $Y_{H}c_{p}$ \\
            \hline
            $OH$ & $N_{OH}$ & $X_{OH} = N_{OH}$ & $\hat{m}N_{OH}$ & $\frac{\hat{m}}{m_{(i)}}$ & $\hat{c}_{p(OH)}(T_{C})$ & $\frac{\hat{c}_{p}}{\hat{m}}$ & $Y_{OH}c_{p}$ \\
            \hline
        \end{tabular}
    \end{center}

    Now we can calculate $c_{p}$, we sum the $Y_{(i)}c_{p}$ values for each species.

    \begin{equation}
        c_{p} = \sum_{i} Y_{(i)}c_{p}
    \end{equation}

    \begin{equation}
        c_{p} = Y_{H_{2}O}c_{p} + Y_{H_{2}}c_{p} + Y_{H}c_{p} + Y_{OH}c_{p}
    \end{equation}

    We perform now calulate the gas constant $R$.

    \begin{equation}
        R = \frac{\hat{R}}{m_{Total}}
    \end{equation}

    \begin{equation}
        c_{v} = c_{p} - R
    \end{equation}

    \begin{equation}
        \gamma = \frac{c_{p}}{c_{p} - \frac{\hat{R}}{\sum_{i} X_{(i)}m_{(i)}}} = \frac{c_{p}}{c_{p}- \frac{\hat{R}}{m_{Total}}} = \frac{c_{p}}{c_{p}- R} = \frac{c_{p}}{c_{v}}
    \end{equation}
    
    \subsection{Nozzle Performance}
    We begin by defining the Area Ratio:

    \begin{equation}
        A_{r} = \frac{\pi R_{e}^{2}}{\pi R_{t}^{2}}
    \end{equation}

    Allowing for us to solve for the Mach number:

    \begin{equation}
        A_{r} = \left(\frac{\gamma + 1}{2}\right)^{-\frac{\gamma + 1}{2(\gamma - 1)}} \frac{1}{M_{e}} \left[1 + \frac{\gamma - 1}{2} M_{e}^{2}\right]^{\frac{\gamma + 1}{2(\gamma - 1)}}
    \end{equation}

    We can solve for Mach number $M_{e}$ by making use of the \verb|scipy.optimize.fsolve| function, which returns the roots 
    of a non-linear function. After solving for $M_{e}$ we can calculate the exit velocity $v_{e}$, exit Pressure $p_{e}$ and exit Temperature $T_{e}$.

    \begin{equation}
        \label{eq:exit-temperature}
        T_{e} = T_{c} \left[1 + \frac{\gamma - 1}{2} M_{e}^{2}\right]^{-1} 
    \end{equation}

    \begin{equation}
        \label{eq:exit-pressure}
        p_{e} = p_{c} \left[1 + \frac{\gamma - 1}{2} M_{e}^{2}\right]^{-\frac{\gamma}{\gamma - 1}}
    \end{equation}

    \begin{equation}
        \label{eq:exit-velocity}
        v_{e} = M_{e} \sqrt{\gamma R T_{e}}
    \end{equation}

    After solving for (Eq. \ref{eq:exit-velocity}), (Eq. \ref{eq:exit-pressure}) and (Eq. \ref{eq:exit-temperature}) we can calculate the mass flow rate $\dot{m}$.

    \begin{equation}
        \label{eq:mass-flow-rate}
        \dot{m} = \frac{A_{t} p_{c} \sqrt{\gamma}}{\sqrt{R T_{c}}} \left[ \frac{2}{\gamma + 1} \right]^{\frac{\gamma + 1}{2(\gamma - 1)}}
    \end{equation}

    Then once you have the mass flow rate you can calculate the thrust $\mathcal{T}$, specific impulse $I_{sp}$ and the thrust coefficient $C_{T}$.

    \begin{equation}
        \label{eq:thrust}
        \mathcal{T} = \dot{m} v_{e} + p_{e} A_{e} - p_{c} A_{t}
    \end{equation}

    \begin{equation}
        \label{eq:specific-impulse}
        I_{sp} = \frac{\mathcal{T}}{\dot{m} g_{0}}
    \end{equation}

    \begin{equation}
        \label{eq:thrust-coefficient}
        C_{T} = \frac{\mathcal{T}}{p_{c} A_{t}}
    \end{equation}

    




    \appendix{}
    \section{Code listings}
    \subsection{SSMEMain.py}
    \lstinputlisting[language=python]{../SSMEMain.py}
    \subsection{exchanger.py}
    \lstinputlisting[language=python]{../exchanger.py}
    \subsection{Iterater.py}
    \lstinputlisting[language=python]{../Iterater.py}


\end{document}